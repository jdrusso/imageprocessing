% This text is proprietary.
% It's a part of presentation made by myself.
% It may not used commercial.
% The noncommercial use such as private and study is free
% Nov. 2006
% Author: Sascha Frank 
% University Freiburg 
% www.informatik.uni-freiburg.de/~frank/
%
% additional usepackage{beamerthemeshadow} is used
%  
%  \beamersetuncovermixins{\opaqueness<1>{25}}{\opaqueness<2->{15}}
%  with this the elements which were coming soon were only hinted
\documentclass{beamer}
%\usepackage{beamerthemesplit}
\usepackage{asymptote}
\usepackage[update,prepend]{epstopdf}
\usepackage{subfigure}
%\usepackage{cite}
%\usepackage[cmex10]{amsmath}
\usepackage{amsfonts}
\usepackage{url}
\usepackage{pgf}
\usepackage{tikz}
\usetikzlibrary{arrows,chains,matrix,positioning,scopes}
%
\makeatletter
\tikzset{join/.code=\tikzset{after node path={%
\ifx\tikzchainprevious\pgfutil@empty\else(\tikzchainprevious)%
edge[every join]#1(\tikzchaincurrent)\fi}}}
\makeatother
%
\tikzset{>=stealth',every on chain/.append style={join},
         every join/.style={->}}
\tikzstyle{labeled}=[execute at begin node=$\scriptstyle,
   execute at end node=$]
\usepackage[style=ieee]{biblatex}
\addbibresource{presentation}
%\ExecuteBibliographyOptions{numeric-comp}

\def\docversion{Notes Version 0.1}

\usetheme{Luebeck}
\usecolortheme{lily}
\usefonttheme{serif}
\title{Sensors for FRC}  
\author{Dr. Travis W. Axtell \\ \emph{travis.axtell@gmail.com}}
\institute{Mentor for FRC Team 612 --- Chantilly, VA \\ 
\quad\quad\quad\, FRC Team 
2035 --- 
Carmel, CA \\ \quad\quad\quad\quad\quad\quad FRC Team 5104 --- Pacific Grove, 
CA}
\date{\docversion\\ \today} 

\addtobeamertemplate{footnote}{}{\vspace{2ex}}

\setbeamercolor{block title}{use=structure,fg=white,bg=blue!75!black}
\setbeamercolor{block body}{use=structure,fg=black,bg=black!20!white}
\setbeamertemplate{blocks}[rounded][shadow=true]
%\setbeamertemplate{blocks}

\begin{document}

\frame{\titlepage} 

\frame{\frametitle{Table of contents}\tableofcontents} 


\section{Introduction} 
\frame{ 
\frametitle{How to read these notes}
\begin{itemize}
\item Underlined terms are concepts you should strive to understand, with the 
goal of being able to explain the concepts to others.
\item Investigate the meaning of new math notation.  Wikipedia can help --- see 
\url{https://en.wikipedia.org/wiki/List_of_mathematical_symbols}.
\item I am available for discussions when you need help.
\end{itemize}
}

\subsection{Robot Hardware}
\frame{
\frametitle{RobotRIO (the ``robot brain'')}
\includegraphics[width=\textwidth]{RobotRio}
}

\frame{
\frametitle{Old hardware connection options (for comparison only)}
\includegraphics[width=2.5in]{DigitalSideCar}
\includegraphics[width=2in]{AnalogSideCar}
}

\subsection{Definitions}
\frame{\frametitle{Definitions}
An \underline{input} signal is a signal generated on some external hardware 
that is going into the RobotRIO.
\\
An \underline{output} signal is a signal generated on the RobotRIO that is 
going to external hardware.
\\
A \underline{digital input/output} port (DIO) on the RobotRIO can perform 
either 
input or output, but not both.
}




\frame{\frametitle{Analog signals}
\begin{figure}
\centering
\includegraphics[width=2in]{analog}
\end{figure}
A \underline{analog signal} represents a varying quantity using a continuous 
independent variable, such as time.
\\
Examples: acoustic, acceleration, velocity, pressure, temperature --- many 
examples exist in nature.
}

\frame{\frametitle{Digital signals}
\includegraphics[width=2in]{pwm} 
\hspace{0.2cm}\includegraphics[width=1.75in]{digital}
\\
A \underline{digital signal} can be represented in two ways:
\begin{enumerate}
\item a continuous-time \underline{waveform} representing a bit stream 
(transitions occur at discrete times)
\item a discrete-time \underline{pulse train} of discrete values
\end{enumerate}
}

\frame{\frametitle{PWM signal}
\begin{figure}
\includegraphics[width=2in]{pwm2}
\end{figure}

A \underline{pulse-width modulated} (PWM) signal \underline{encodes} a value 
into a pulse width $\tau$ of a signal with periodicity $T$.
}

\frame{\frametitle{Definitions}
Noise is random changes to the signal due to physical properties of the sensor.
All sensors are subject to noise.  Noise can only be quantified by statistics, 
as its actual value $n(t)$ cannot be known.

It is desirable for a control system to be \underline{Robust} to noise, meaning 
that the noise does not strongly affect the output.

\underline{Uncertainty} is the incorrect (or absent) modeling of a system.

}




\section{Sensors}
\subsection{Simple}
\frame{
\frametitle{Contact switch}
\begin{figure}
\includegraphics[width=2in]{LimitSwitch}
\end{figure}

We can treat this sensor like a digital signal.
\underline{Normally Open} (NO) is 0 at rest, 1 when pressed.  
\underline{Normally Closed} (NC) is 1 at rest, 0 when pressed.  Only need 2 
wires (ground and signal), because the \underline{Common} (COM) is connected to 
either NC or NO, but not both.

Difficulty: metal bends easily.
}

\frame{
\frametitle{Electronic switch}
\includegraphics[width=\textwidth]{DriverStation}
}

\frame{
\frametitle{Pressure switch}
\includegraphics[width=2in]{PressureSwitch}
}

\frame{
\frametitle{Potentiometer}
\includegraphics[width=1in]{Potentiometer}
\includegraphics[width=1in]{Potentiometer2}
}

\frame{
\frametitle{Rotary encoder}
\includegraphics[width=1in]{OpticalEncoder}
\includegraphics[width=1in]{OpticalEncoder2}
\includegraphics[width=1in]{MagneticEncoder}
\includegraphics[width=1in]{MagneticEncoder2}
}

\subsection{Imaging}
\frame{
\frametitle{AXIS Webcam}
\includegraphics[width=1in]{AXISwebcam}
}

\frame{
\frametitle{PixyCam}
\includegraphics[width=1in]{PixyCam}
}

\frame{
\frametitle{Kinect}
\includegraphics[width=\textwidth]{kinect}
}

\frame{
\frametitle{Kinect}
\includegraphics[width=1in]{kinect}
}

\frame{
\frametitle{Playstation Camera}
\includegraphics[width=1in]{PlaystationCamera}
}

\subsection{Location and Orientation}
\frame{
\frametitle{Gyro}
\includegraphics[width=1in]{Gyro}
}

\frame{
\frametitle{Accelerometer}
}

\frame{
\frametitle{GPS}
}

\frame{
\frametitle{Indoor GPS}
}

\subsection{Distance estimation}
\frame{
\frametitle{Ultrasonic rangefinder}
\includegraphics[width=1in]{Ultrasonic}
\includegraphics[width=1in]{Ultrasonic2}
}

\frame{
\frametitle{Laser rangefinder}
\includegraphics[width=1in]{LaserRangefinder}
}

\section{Haptic and Controls}
\frame{
\frametitle{Vibrating joystick}
}

\frame{
\frametitle{Wii controller Wiimote}
}

\frame{
\frametitle{Wii balance board}
}

\section{Conclusion}
\frame{
\frametitle{LEGO Mindstorm}
The LEGO Mindstorm has equivalent sensors for:
Color, light detection, axis rotation, contact, IR, ultrasonic, ...
}

\subsection{Resources}
\frame{ 
\frametitle{Resources}
[1] \href{http://www}{\underline{TBD}}
}

\end{document}
