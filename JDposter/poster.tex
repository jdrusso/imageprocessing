\documentclass{beamer}
\usepackage[orientation=landscape, size=a0, scale=2]{beamerposter}
\usepackage{amsmath,amsthm, amssymb, latexsym}
\usepackage[absolute,overlay]{textpos}
\usepackage{tikz}
\usepackage{pgf}
\usepackage{epstopdf}
%\mode<presentation>{\usetheme{Luebeck}}
\mode<presentation>{\usetheme{I6pd2}}
\usecolortheme{lily}
\usefonttheme{serif}
\setbeamercolor{block title}{use=structure,fg=white,bg=blue!75!black}
\setbeamercolor{block body}{use=structure,fg=black,bg=black!20!white}
\setbeamertemplate{blocks}[rounded][shadow=true]

\author{Maggie Graves \& JD Russo}
\title{Applications of Linear Algebra in Neural Networks}
\institute{MATH 245: Linear Algebra\\
 Department of Mathematics, Bucknell University, Lewisburg, PA}
\makeatletter
\let\Author\@author
\let\Title\@title
\makeatother

\begin{document}

%\makeatletter
%
%\pgfdeclareshape{rectangle with diagonal fill}
%{
%    % This bit from \pgflibarayshapes.code.tex
%    \inheritsavedanchors[from=rectangle]
%    \inheritanchorborder[from=rectangle]
%    \inheritanchor[from=rectangle]{north}
%    \inheritanchor[from=rectangle]{north west}
%    \inheritanchor[from=rectangle]{north east}
%    \inheritanchor[from=rectangle]{center}
%    \inheritanchor[from=rectangle]{west}
%    \inheritanchor[from=rectangle]{east}
%    \inheritanchor[from=rectangle]{mid}
%    \inheritanchor[from=rectangle]{mid west}
%    \inheritanchor[from=rectangle]{mid east}
%    \inheritanchor[from=rectangle]{base}
%    \inheritanchor[from=rectangle]{base west}
%    \inheritanchor[from=rectangle]{base east}
%    \inheritanchor[from=rectangle]{south}
%    \inheritanchor[from=rectangle]{south west}
%    \inheritanchor[from=rectangle]{south east}
%
%    \inheritbackgroundpath[from=rectangle]
%    \inheritbeforebackgroundpath[from=rectangle]
%    \inheritbehindforegroundpath[from=rectangle]
%    \inheritforegroundpath[from=rectangle]
%    \inheritbeforeforegroundpath[from=rectangle]
%
%   % Now do the background filling.
%    \behindbackgroundpath{%
%        % \southwest and \northeast defined by rectangle, but
%        % (somewhat annoyingly) not \southeast and \northwest
%        % so use this workaround.
%        \pgfextractx{\pgf@xa}{\southwest}%
%        \pgfextracty{\pgf@ya}{\southwest}%
%        \pgfextractx{\pgf@xb}{\northeast}%
%        \pgfextracty{\pgf@yb}{\northeast}%
%        \ifpgf@diagonal@lefttoright
%            \def\pgf@diagonal@point@a{\pgfpoint{\pgf@xa}{\pgf@yb}}%
%            \def\pgf@diagonal@point@b{\pgfpoint{\pgf@xb}{\pgf@ya}}%
%        \else
%            \def\pgf@diagonal@point@a{\southwest}%
%            \def\pgf@diagonal@point@b{\northeast}%
%        \fi
%        \pgfpathmoveto{\pgf@diagonal@point@a}%
%        \pgfpathlineto{\northeast}%
%        \pgfpathlineto{\pgfpoint{\pgf@xb}{\pgf@ya}}%
%        \pgfpathclose
%        \ifpgf@diagonal@lefttoright
%            \color{\pgf@diagonal@top@color}%
%        \else
%            \color{\pgf@diagonal@bottom@color}%
%        \fi
%        \pgfusepath{fill}%
%        \pgfpathmoveto{\pgfpoint{\pgf@xa}{\pgf@yb}}%
%        \pgfpathlineto{\southwest}%
%        \pgfpathlineto{\pgf@diagonal@point@b}%
%        \pgfpathclose
%        \ifpgf@diagonal@lefttoright
%            \color{\pgf@diagonal@bottom@color}%
%        \else
%            \color{\pgf@diagonal@top@color}%
%        \fi
%        \pgfusepath{fill}%
%    }
%}
%
%\newif\ifpgf@diagonal@lefttoright
%\def\pgf@diagonal@top@color{white}
%\def\pgf@diagonal@bottom@color{gray!30}
%
%% Use these with PGF
%\def\pgfsetdiagonaltopcolor#1{\def\pgf@diagonal@top@color{#1}}%
%\def\pgfsetdiagonalbottomcolor#1{\def\pgf@diagonal@bottom@color{#1}}%
%\def\pgfsetdiagonallefttoright{\pgf@diagonal@lefttorighttrue}%
%\def\pgfsetdiagonalrighttoleft{\pgf@diagonal@lefttorightfalse}%
%
%% Use these with TikZ
%\tikzoption{diagonal top color}{\pgfsetdiagonaltopcolor{#1}}
%\tikzoption{diagonal bottom color}{\pgfsetdiagonalbottomcolor{#1}}
%\tikzoption{diagonal from left to right}[]{\pgfsetdiagonallefttoright}
%\tikzoption{diagonal from right to left}[]{\pgfsetdiagonalrighttoleft}
%
%\makeatother
\begin{frame}
%\makeatletter

%\makeatother

\vspace{15cm}

%\begin{tikzpicture}[inner sep=6pt,scale=2]
%    \node[shape=rectangle with diagonal fill, diagonal from right to left, 
%          draw] at (1,1) {\Large 1};
%\end{tikzpicture}

\begin{columns}
\begin{column}{0.3\textwidth}
\begin{block}{Title of block 1}
First block
\end{block}
\begin{block}{Title of block 2}
Second block
\end{block}

\end{column}
\begin{column}{0.3\textwidth}
\begin{block}{Title of block 3}
Third block
\end{block}
\begin{block}{Title of block 4}
Fourth block
\end{block}
\end{column}
\begin{column}{0.3\textwidth}
content...
\end{column}
\end{columns}
\end{frame}

\end{document}